% -------------------------------------------------------------------------
% Setup
% -------------------------------------------------------------------------
\documentclass[11pt, aspectratio=149]{beamer}
% Options for aspectratio: 1610, 149, 54, 43 and 32, 169
\usepackage[utf8]{inputenc}
\usepackage[english]{babel}% Alternative: 'norsk'
\usepackage[expansion=false]{microtype}% Fixes to make typography better
\usecolortheme{beaver} % Decent options: beaver, rose, crane
\usepackage{listings}% To include source-code
\usepackage{booktabs}% Professional tables
\usefonttheme{serif}
\usepackage{mathptmx}
\usepackage[scaled=0.9]{helvet}
\usepackage{courier}

\title{Introduction to Kernel Density Estimation}
\subtitle{A graphical tutorial}
%\institute{PåVei 2018, UiB }
\date{\today}
\author{tommyod @ GitHub}

% -------------------------------------------------------------------------
% Package imports
% -------------------------------------------------------------------------
\usepackage{etoolbox}
\usepackage{graphicx}
\usepackage{tikz}
\usepackage{amsmath}
\usepackage{amsthm}
\usepackage{amsfonts}
\usepackage{amssymb}
\usepackage{mathtools}
\usepackage{graphicx}
\usepackage{hyperref}
\usepackage{listings}
\usepackage[sharp]{easylist}
\usepackage{multicol}
\usepackage{tikz-cd}

\usefonttheme{professionalfonts}
\usepackage{fontspec}
\setmainfont{Open Sans}
\setsansfont{Open Sans}
\setmonofont{Ubuntu Mono}
\usefonttheme{serif}

%gets rid of bottom navigation bars
\setbeamertemplate{footline}[frame number]{}

%gets rid of bottom navigation symbols
\setbeamertemplate{navigation symbols}{}

% Set up colors to be used
\definecolor{purered}{RGB}{31,119,180}
\definecolor{titlered}{RGB}{31,119,180}
\definecolor{bggray}{RGB}{242,242,242}
\definecolor{bggraydark}{RGB}{217,217,217}

% Change the default colors

\setbeamercolor*{title}{bg=bggray,fg=titlered}
\AtBeginEnvironment{theorem}{%
	\setbeamercolor{block title}{fg=titlered, bg=bggraydark}
	\setbeamercolor{block body}{fg=black,bg=bggray}
}
\AtBeginEnvironment{proof}{%
	\setbeamercolor{block title}{bg=bggraydark}
	\setbeamercolor{block body}{fg=black,bg=bggray}
}
\AtBeginEnvironment{example}{%
	\setbeamercolor{block title example}{bg=bggraydark}
	\setbeamercolor{block body example}{fg=black,bg=bggray}
}
\AtBeginEnvironment{definition}{%
	\setbeamercolor{block title}{bg=bggraydark}
	\setbeamercolor{block body}{fg=black,bg=bggray}
}

\setbeamercolor{block title example}{bg=bggraydark}
\setbeamercolor{block body example}{fg=black,bg=bggray}
\setbeamercolor{block title}{bg=bggraydark}
\setbeamercolor{block body}{fg=black,bg=bggray}

\setbeamercolor{frametitle}{fg=titlered,bg=bggray}
\setbeamercolor{section in head/foot}{bg=black}
\setbeamercolor{author in head/foot}{bg=black}
\setbeamercolor{date in head/foot}{fg=titlered}


% Spacing for lsits
\newcommand{\listSpace}{0.4em}

% Theorems, equations, definitions setup
\theoremstyle{plain}

\usepackage{etoolbox}
\usepackage{lipsum}

\makeatletter
\patchcmd{\beamer@sectionintoc}
{\vfill}
{\vskip\itemsep}
{}
{}
\makeatother  

\AtBeginSection[]{
	\begin{frame}
		\vfill
		\centering
		\begin{beamercolorbox}[sep=8pt,center,shadow=false,rounded=false]{title}
			\usebeamerfont{title}\insertsectionhead\par%
		\end{beamercolorbox}
		\vfill
	\end{frame}
}

% -------------------------------------------------------------------------
% Document start
% -------------------------------------------------------------------------
\begin{document}
\maketitle

% -------------------------------------------------------------------------

\section{Introduction}

% -------------------------------------------------------------------------
\begin{frame}[fragile, t]{What is a kernel density estimate?}
	On every data point $x_i$, we place a kernel function $K$.
	The kernel density estimate is 
	\begin{equation*}
	\hat{f}(x) = \frac{1}{N} \sum_{i=1}^{N} K(x - x_i).
	\end{equation*}
	\vfill
	\begin{figure}
		\centering
		\includegraphics[width=0.8\linewidth]{figs/intro_kde_gaussian_1}
	\end{figure}
\end{frame}


\begin{frame}[fragile, t]{What is a kernel density estimate?}
	On every data point $x_i$, we place a kernel function $K$.
	The kernel density estimate is 
	\begin{equation*}
	\hat{f}(x) = \frac{1}{N} \sum_{i=1}^{N} K(x - x_i).
	\end{equation*}
	\vfill
	\begin{figure}
		\centering
		\includegraphics[width=0.8\linewidth]{figs/intro_kde_gaussian_2}
	\end{figure}
\end{frame}


\begin{frame}[fragile, t]{What is a kernel density estimate?}
	On every data point $x_i$, we place a kernel function $K$.
	The kernel density estimate is 
	\begin{equation*}
	\hat{f}(x) = \frac{1}{N} \sum_{i=1}^{N} K(x - x_i).
	\end{equation*}
	\vfill
	\begin{figure}
		\centering
		\includegraphics[width=0.8\linewidth]{figs/intro_kde_gaussian_3}
	\end{figure}
\end{frame}


\begin{frame}[fragile, t]{What is a kernel density estimate?}
	On every data point $x_i$, we place a kernel function $K$.
	The kernel density estimate is 
	\begin{equation*}
	\hat{f}(x) = \frac{1}{N} \sum_{i=1}^{N} K(x - x_i).
	\end{equation*}
	\vfill
	\begin{figure}
		\centering
		\includegraphics[width=0.8\linewidth]{figs/intro_kde_gaussian_4}
	\end{figure}
\end{frame}


\begin{frame}[fragile, t]{What is a kernel density estimate?}
	On every data point $x_i$, we place a kernel function $K$.
	The kernel density estimate is 
	\begin{equation*}
	\hat{f}(x) = \frac{1}{N} \sum_{i=1}^{N} K(x - x_i).
	\end{equation*}
	\vfill
	\begin{figure}
		\centering
		\includegraphics[width=0.8\linewidth]{figs/intro_kde_gaussian_5}
	\end{figure}
\end{frame}


\begin{frame}[fragile, t]{What is a kernel density estimate?}
	On every data point $x_i$, we place a kernel function $K$.
	The kernel density estimate is 
	\begin{equation*}
	\hat{f}(x) = \frac{1}{N} \sum_{i=1}^{N} K(x - x_i).
	\end{equation*}
	\vfill
	\begin{figure}
		\centering
		\includegraphics[width=0.8\linewidth]{figs/intro_kde_gaussian_6}
	\end{figure}
\end{frame}


\begin{frame}[fragile, t]{What is a kernel density estimate?}
	On every data point $x_i$, we place a kernel function $K$.
	The kernel density estimate is 
	\begin{equation*}
	\hat{f}(x) = \frac{1}{N} \sum_{i=1}^{N} K(x - x_i).
	\end{equation*}
	\vfill
	\begin{figure}
		\centering
		\includegraphics[width=0.8\linewidth]{figs/intro_kde_gaussian_7}
	\end{figure}
\end{frame}


\begin{frame}[fragile, t]{What is a kernel density estimate?}
	On every data point $x_i$, we place a kernel function $K$.
	The kernel density estimate is 
	\begin{equation*}
	\hat{f}(x) = \frac{1}{N} \sum_{i=1}^{N} K(x - x_i).
	\end{equation*}
	\vfill
	\begin{figure}
		\centering
		\includegraphics[width=0.8\linewidth]{figs/intro_kde_gaussian_8}
	\end{figure}
\end{frame}


\begin{frame}[fragile, t]{What is a kernel density estimate?}
	On every data point $x_i$, we place a kernel function $K$.
	The kernel density estimate is 
	\begin{equation*}
	\hat{f}(x) = \frac{1}{N} \sum_{i=1}^{N} K(x - x_i).
	\end{equation*}
	\vfill
	\begin{figure}
		\centering
		\includegraphics[width=0.8\linewidth]{figs/intro_kde_gaussian_9}
	\end{figure}
\end{frame}


\begin{frame}[fragile, t]{What is a kernel density estimate?}
	On every data point $x_i$, we place a kernel function $K$.
	The kernel density estimate is 
	\begin{equation*}
	\hat{f}(x) = \frac{1}{N} \sum_{i=1}^{N} K(x - x_i).
	\end{equation*}
	\vfill
	\begin{figure}
		\centering
		\includegraphics[width=0.8\linewidth]{figs/intro_kde_gaussian_10}
	\end{figure}
\end{frame}


\begin{frame}[fragile, t]{What is a kernel density estimate?}
	On every data point $x_i$, we place a kernel function $K$.
	The kernel density estimate is 
	\begin{equation*}
	\hat{f}(x) = \frac{1}{N} \sum_{i=1}^{N} K(x - x_i).
	\end{equation*}
	\vfill
	\begin{figure}
		\centering
		\includegraphics[width=0.8\linewidth]{figs/intro_kde_gaussian_11}
	\end{figure}
\end{frame}


% -------------------------------------------------------------------------
\begin{frame}[fragile, t]{Choice of kernel}
	The kernel function $K$ is typically
	\vspace{1em}
	\begin{easylist}[itemize]
		\ListProperties(Space=\listSpace, Space*=\listSpace)
		# everywhere non-negative: $K(x) \geq 0$ for every $x$
		# symmetric: $K(x) = K(-x)$  for every $x$
		# decreasing: $K'(x) \leq 0$  for every $x > 0$.
	\end{easylist}
	\vfill
	\begin{figure}
		\centering
		\includegraphics[width=1\linewidth]{figs/kde_intro_kernels}
	\end{figure}
\end{frame}


% -------------------------------------------------------------------------
\begin{frame}[fragile, t]{Choice of kernel}
	The \emph{triangular} kernel (or \emph{linear} kernel) is given by
	\begin{equation*}
	f(x) \propto \max(1 - |x|, 0).
	\end{equation*}
	\vfill
	\begin{figure}
		\centering
		\includegraphics[width=0.8\linewidth]{figs/intro_kde_tri_1}
	\end{figure}
\end{frame}


\begin{frame}[fragile, t]{Choice of kernel}
	The \emph{triangular} kernel (or \emph{linear} kernel) is given by
	\begin{equation*}
	f(x) \propto \max(1 - |x|, 0).
	\end{equation*}
	\vfill
	\begin{figure}
		\centering
		\includegraphics[width=0.8\linewidth]{figs/intro_kde_tri_2}
	\end{figure}
\end{frame}


\begin{frame}[fragile, t]{Choice of kernel}
	The \emph{triangular} kernel (or \emph{linear} kernel) is given by
	\begin{equation*}
	f(x) \propto \max(1 - |x|, 0).
	\end{equation*}
	\vfill
	\begin{figure}
		\centering
		\includegraphics[width=0.8\linewidth]{figs/intro_kde_tri_3}
	\end{figure}
\end{frame}


\begin{frame}[fragile, t]{Choice of kernel}
	The \emph{triangular} kernel (or \emph{linear} kernel) is given by
	\begin{equation*}
	f(x) \propto \max(1 - |x|, 0).
	\end{equation*}
	\vfill
	\begin{figure}
		\centering
		\includegraphics[width=0.8\linewidth]{figs/intro_kde_tri_4}
	\end{figure}
\end{frame}


\begin{frame}[fragile, t]{Choice of kernel}
	The \emph{triangular} kernel (or \emph{linear} kernel) is given by
	\begin{equation*}
	f(x) \propto \max(1 - |x|, 0).
	\end{equation*}
	\vfill
	\begin{figure}
		\centering
		\includegraphics[width=0.8\linewidth]{figs/intro_kde_tri_5}
	\end{figure}
\end{frame}


\begin{frame}[fragile, t]{Choice of kernel}
	The \emph{triangular} kernel (or \emph{linear} kernel) is given by
	\begin{equation*}
	f(x) \propto \max(1 - |x|, 0).
	\end{equation*}
	\vfill
	\begin{figure}
		\centering
		\includegraphics[width=0.8\linewidth]{figs/intro_kde_tri_6}
	\end{figure}
\end{frame}


\begin{frame}[fragile, t]{Choice of kernel}
	The \emph{triangular} kernel (or \emph{linear} kernel) is given by
	\begin{equation*}
	f(x) \propto \max(1 - |x|, 0).
	\end{equation*}
	\vfill
	\begin{figure}
		\centering
		\includegraphics[width=0.8\linewidth]{figs/intro_kde_tri_7}
	\end{figure}
\end{frame}


\begin{frame}[fragile, t]{Choice of kernel}
	The \emph{triangular} kernel (or \emph{linear} kernel) is given by
	\begin{equation*}
	f(x) \propto \max(1 - |x|, 0).
	\end{equation*}
	\vfill
	\begin{figure}
		\centering
		\includegraphics[width=0.8\linewidth]{figs/intro_kde_tri_8}
	\end{figure}
\end{frame}


\begin{frame}[fragile, t]{Choice of kernel}
	The \emph{triangular} kernel (or \emph{linear} kernel) is given by
	\begin{equation*}
	f(x) \propto \max(1 - |x|, 0).
	\end{equation*}
	\vfill
	\begin{figure}
		\centering
		\includegraphics[width=0.8\linewidth]{figs/intro_kde_tri_9}
	\end{figure}
\end{frame}


\begin{frame}[fragile, t]{Choice of kernel}
	The \emph{triangular} kernel (or \emph{linear} kernel) is given by
	\begin{equation*}
	f(x) \propto \max(1 - |x|, 0).
	\end{equation*}
	\vfill
	\begin{figure}
		\centering
		\includegraphics[width=0.8\linewidth]{figs/intro_kde_tri_10}
	\end{figure}
\end{frame}


\begin{frame}[fragile, t]{Choice of kernel}
	The \emph{triangular} kernel (or \emph{linear} kernel) is given by
	\begin{equation*}
	f(x) \propto \max(1 - |x|, 0).
	\end{equation*}
	\vfill
	\begin{figure}
		\centering
		\includegraphics[width=0.8\linewidth]{figs/intro_kde_tri_11}
	\end{figure}
\end{frame}





% -------------------------------------------------------------------------
\begin{frame}[fragile, t]{Choice of bandwidth}
	We use $h$ to control for the \emph{bandwidth} of $\hat{f}(x)$ by writing
	\begin{equation*}
	\hat{f}(x) = \frac{1}{Nh} \sum_{i=1}^{N} K\left(\frac{x - x_i}{h}\right).
	\end{equation*}
	\vfill
	\begin{figure}
		\centering
		\includegraphics[width=0.8\linewidth]{figs/intro_kde_bw_1}
	\end{figure}
\end{frame}


\begin{frame}[fragile, t]{Choice of bandwidth}
	We use $h$ to control for the \emph{bandwidth} of $\hat{f}(x)$ by writing
	\begin{equation*}
	\hat{f}(x) = \frac{1}{Nh} \sum_{i=1}^{N} K\left(\frac{x - x_i}{h}\right).
	\end{equation*}
	\vfill
	\begin{figure}
		\centering
		\includegraphics[width=0.8\linewidth]{figs/intro_kde_bw_2}
	\end{figure}
\end{frame}


\begin{frame}[fragile, t]{Choice of bandwidth}
	We use $h$ to control for the \emph{bandwidth} of $\hat{f}(x)$ by writing
	\begin{equation*}
	\hat{f}(x) = \frac{1}{Nh} \sum_{i=1}^{N} K\left(\frac{x - x_i}{h}\right).
	\end{equation*}
	\vfill
	\begin{figure}
		\centering
		\includegraphics[width=0.8\linewidth]{figs/intro_kde_bw_3}
	\end{figure}
\end{frame}


\begin{frame}[fragile, t]{Choice of bandwidth}
	We use $h$ to control for the \emph{bandwidth} of $\hat{f}(x)$ by writing
	\begin{equation*}
	\hat{f}(x) = \frac{1}{Nh} \sum_{i=1}^{N} K\left(\frac{x - x_i}{h}\right).
	\end{equation*}
	\vfill
	\begin{figure}
		\centering
		\includegraphics[width=0.8\linewidth]{figs/intro_kde_bw_4}
	\end{figure}
\end{frame}


\begin{frame}[fragile, t]{Choice of bandwidth}
	We use $h$ to control for the \emph{bandwidth} of $\hat{f}(x)$ by writing
	\begin{equation*}
	\hat{f}(x) = \frac{1}{Nh} \sum_{i=1}^{N} K\left(\frac{x - x_i}{h}\right).
	\end{equation*}
	\vfill
	\begin{figure}
		\centering
		\includegraphics[width=0.8\linewidth]{figs/intro_kde_bw_5}
	\end{figure}
\end{frame}


\begin{frame}[fragile, t]{Choice of bandwidth}
	We use $h$ to control for the \emph{bandwidth} of $\hat{f}(x)$ by writing
	\begin{equation*}
	\hat{f}(x) = \frac{1}{Nh} \sum_{i=1}^{N} K\left(\frac{x - x_i}{h}\right).
	\end{equation*}
	\vfill
	\begin{figure}
		\centering
		\includegraphics[width=0.8\linewidth]{figs/intro_kde_bw_6}
	\end{figure}
\end{frame}


\begin{frame}[fragile, t]{Choice of bandwidth}
	We use $h$ to control for the \emph{bandwidth} of $\hat{f}(x)$ by writing
	\begin{equation*}
	\hat{f}(x) = \frac{1}{Nh} \sum_{i=1}^{N} K\left(\frac{x - x_i}{h}\right).
	\end{equation*}
	\vfill
	\begin{figure}
		\centering
		\includegraphics[width=0.8\linewidth]{figs/intro_kde_bw_7}
	\end{figure}
\end{frame}


\begin{frame}[fragile, t]{Choice of bandwidth}
	We use $h$ to control for the \emph{bandwidth} of $\hat{f}(x)$ by writing
	\begin{equation*}
	\hat{f}(x) = \frac{1}{Nh} \sum_{i=1}^{N} K\left(\frac{x - x_i}{h}\right).
	\end{equation*}
	\vfill
	\begin{figure}
		\centering
		\includegraphics[width=0.8\linewidth]{figs/intro_kde_bw_8}
	\end{figure}
\end{frame}


\begin{frame}[fragile, t]{Choice of bandwidth}
	We use $h$ to control for the \emph{bandwidth} of $\hat{f}(x)$ by writing
	\begin{equation*}
	\hat{f}(x) = \frac{1}{Nh} \sum_{i=1}^{N} K\left(\frac{x - x_i}{h}\right).
	\end{equation*}
	\vfill
	\begin{figure}
		\centering
		\includegraphics[width=0.8\linewidth]{figs/intro_kde_bw_9}
	\end{figure}
\end{frame}


\begin{frame}[fragile, t]{Choice of bandwidth}
	We use $h$ to control for the \emph{bandwidth} of $\hat{f}(x)$ by writing
	\begin{equation*}
	\hat{f}(x) = \frac{1}{Nh} \sum_{i=1}^{N} K\left(\frac{x - x_i}{h}\right).
	\end{equation*}
	\vfill
	\begin{figure}
		\centering
		\includegraphics[width=0.8\linewidth]{figs/intro_kde_bw_10}
	\end{figure}
\end{frame}





% -------------------------------------------------------------------------
\begin{frame}[fragile, t]{Choice of bandwidth - Silverman}
	\emph{Silverman's rule of thumb} computes an optimal $h$ by assuming that the data is normally distributed. Good starting point in many cases.
	\vfill
	\begin{figure}
		\centering
		\includegraphics[width=0.8\linewidth]{figs/intro_kde_silverman_1}
	\end{figure}
\end{frame}

\begin{frame}[fragile, t]{Choice of bandwidth - Silverman}
	\emph{Silverman's rule of thumb} computes an optimal $h$ by assuming that the data is normally distributed. Good starting point in many cases.
	\vfill
	\begin{figure}
		\centering
		\includegraphics[width=0.8\linewidth]{figs/intro_kde_silverman_2}
	\end{figure}
\end{frame}

\begin{frame}[fragile, t]{Choice of bandwidth - Silverman}
	\emph{Silverman's rule of thumb} computes an optimal $h$ by assuming that the data is normally distributed. Good starting point in many cases.
	\vfill
	\begin{figure}
		\centering
		\includegraphics[width=0.8\linewidth]{figs/intro_kde_silverman_3}
	\end{figure}
\end{frame}


% -------------------------------------------------------------------------
\begin{frame}[fragile, t]{Choice of bandwidth - ISJ}
	The \emph{Improved Sheather Jones} (ISJ) algorithm is more robust with respect to \emph{multimodality}.
	\vfill
	\begin{figure}
		\centering
		\includegraphics[width=0.8\linewidth]{figs/intro_kde_ISJ_1}
	\end{figure}
\end{frame}

\begin{frame}[fragile, t]{Choice of bandwidth - ISJ}
	The \emph{Improved Sheather Jones} (ISJ) algorithm is more robust with respect to \emph{multimodality}.
	\vfill
	\begin{figure}
		\centering
		\includegraphics[width=0.8\linewidth]{figs/intro_kde_ISJ_2}
	\end{figure}
\end{frame}

\begin{frame}[fragile, t]{Choice of bandwidth - ISJ}
	The \emph{Improved Sheather Jones} (ISJ) algorithm is more robust with respect to \emph{multimodality}.
	\vfill
	\begin{figure}
		\centering
		\includegraphics[width=0.8\linewidth]{figs/intro_kde_ISJ_3}
	\end{figure}
\end{frame}

\begin{frame}[fragile, t]{Choice of bandwidth - ISJ}
	The \emph{Improved Sheather Jones} (ISJ) algorithm is more robust with respect to \emph{multimodality}.
	\vfill
	\begin{figure}
		\centering
		\includegraphics[width=0.8\linewidth]{figs/intro_kde_ISJ_4}
	\end{figure}
\end{frame}


% -------------------------------------------------------------------------
\begin{frame}[fragile, t]{Weighting data}
	It's possible to add weights $w_i$ to data points $x_i$ by writing
	\begin{equation*}
	\hat{f}(x) = \frac{1}{h} \sum_{i=1}^{N} w_i K\left(\frac{x - x_i}{h}\right), \text{ where } \sum_{i=1}^{N} w_i = 1.
	\end{equation*}
	\vfill
	\begin{figure}
		\centering
		\includegraphics[width=0.8\linewidth]{figs/intro_kde_w_gaussian_1}
	\end{figure}
\end{frame}


\begin{frame}[fragile, t]{Weighting data}
	It's possible to add weights $w_i$ to data points $x_i$ by writing
	\begin{equation*}
	\hat{f}(x) = \frac{1}{h} \sum_{i=1}^{N} w_i K\left(\frac{x - x_i}{h}\right), \text{ where } \sum_{i=1}^{N} w_i = 1.
	\end{equation*}
	\vfill
	\begin{figure}
		\centering
		\includegraphics[width=0.8\linewidth]{figs/intro_kde_w_gaussian_2}
	\end{figure}
\end{frame}


\begin{frame}[fragile, t]{Weighting data}
	It's possible to add weights $w_i$ to data points $x_i$ by writing
	\begin{equation*}
	\hat{f}(x) = \frac{1}{h} \sum_{i=1}^{N} w_i K\left(\frac{x - x_i}{h}\right), \text{ where } \sum_{i=1}^{N} w_i = 1.
	\end{equation*}
	\vfill
	\begin{figure}
		\centering
		\includegraphics[width=0.8\linewidth]{figs/intro_kde_w_gaussian_3}
	\end{figure}
\end{frame}


\begin{frame}[fragile, t]{Weighting data}
	It's possible to add weights $w_i$ to data points $x_i$ by writing
	\begin{equation*}
	\hat{f}(x) = \frac{1}{h} \sum_{i=1}^{N} w_i K\left(\frac{x - x_i}{h}\right), \text{ where } \sum_{i=1}^{N} w_i = 1.
	\end{equation*}
	\vfill
	\begin{figure}
		\centering
		\includegraphics[width=0.8\linewidth]{figs/intro_kde_w_gaussian_4}
	\end{figure}
\end{frame}


\begin{frame}[fragile, t]{Weighting data}
	It's possible to add weights $w_i$ to data points $x_i$ by writing
	\begin{equation*}
	\hat{f}(x) = \frac{1}{h} \sum_{i=1}^{N} w_i K\left(\frac{x - x_i}{h}\right), \text{ where } \sum_{i=1}^{N} w_i = 1.
	\end{equation*}
	\vfill
	\begin{figure}
		\centering
		\includegraphics[width=0.8\linewidth]{figs/intro_kde_w_gaussian_5}
	\end{figure}
\end{frame}


\begin{frame}[fragile, t]{Weighting data}
	It's possible to add weights $w_i$ to data points $x_i$ by writing
	\begin{equation*}
	\hat{f}(x) = \frac{1}{h} \sum_{i=1}^{N} w_i K\left(\frac{x - x_i}{h}\right), \text{ where } \sum_{i=1}^{N} w_i = 1.
	\end{equation*}
	\vfill
	\begin{figure}
		\centering
		\includegraphics[width=0.8\linewidth]{figs/intro_kde_w_gaussian_6}
	\end{figure}
\end{frame}


\begin{frame}[fragile, t]{Weighting data}
	It's possible to add weights $w_i$ to data points $x_i$ by writing
	\begin{equation*}
	\hat{f}(x) = \frac{1}{h} \sum_{i=1}^{N} w_i K\left(\frac{x - x_i}{h}\right), \text{ where } \sum_{i=1}^{N} w_i = 1.
	\end{equation*}
	\vfill
	\begin{figure}
		\centering
		\includegraphics[width=0.8\linewidth]{figs/intro_kde_w_gaussian_7}
	\end{figure}
\end{frame}


\begin{frame}[fragile, t]{Weighting data}
	It's possible to add weights $w_i$ to data points $x_i$ by writing
	\begin{equation*}
	\hat{f}(x) = \frac{1}{h} \sum_{i=1}^{N} w_i K\left(\frac{x - x_i}{h}\right), \text{ where } \sum_{i=1}^{N} w_i = 1.
	\end{equation*}
	\vfill
	\begin{figure}
		\centering
		\includegraphics[width=0.8\linewidth]{figs/intro_kde_w_gaussian_8}
	\end{figure}
\end{frame}


\begin{frame}[fragile, t]{Weighting data}
	It's possible to add weights $w_i$ to data points $x_i$ by writing
	\begin{equation*}
	\hat{f}(x) = \frac{1}{h} \sum_{i=1}^{N} w_i K\left(\frac{x - x_i}{h}\right), \text{ where } \sum_{i=1}^{N} w_i = 1.
	\end{equation*}
	\vfill
	\begin{figure}
		\centering
		\includegraphics[width=0.8\linewidth]{figs/intro_kde_w_gaussian_9}
	\end{figure}
\end{frame}


\begin{frame}[fragile, t]{Weighting data}
	It's possible to add weights $w_i$ to data points $x_i$ by writing
	\begin{equation*}
	\hat{f}(x) = \frac{1}{h} \sum_{i=1}^{N} w_i K\left(\frac{x - x_i}{h}\right), \text{ where } \sum_{i=1}^{N} w_i = 1.
	\end{equation*}
	\vfill
	\begin{figure}
		\centering
		\includegraphics[width=0.8\linewidth]{figs/intro_kde_w_gaussian_10}
	\end{figure}
\end{frame}


\begin{frame}[fragile, t]{Weighting data}
	It's possible to add weights $w_i$ to data points $x_i$ by writing
	\begin{equation*}
	\hat{f}(x) = \frac{1}{h} \sum_{i=1}^{N} w_i K\left(\frac{x - x_i}{h}\right), \text{ where } \sum_{i=1}^{N} w_i = 1.
	\end{equation*}
	\vfill
	\begin{figure}
		\centering
		\includegraphics[width=0.8\linewidth]{figs/intro_kde_w_gaussian_11}
	\end{figure}
\end{frame}


% -------------------------------------------------------------------------
\begin{frame}[fragile, t]{Bounded domains}
	A simple trick to overcome bias at boundaries is to mirror the data.
	This ensures that $\hat{f}'(x) = 0$ at the boundary.
	\vfill
	\begin{figure}
		\centering
		\includegraphics[width=0.8\linewidth]{figs/intro_kde_mirror_1}
	\end{figure}
\end{frame}

\begin{frame}[fragile, t]{Bounded domains}
	A simple trick to overcome bias at boundaries is to mirror the data.
	This ensures that $\hat{f}'(x) = 0$ at the boundary.
	\vfill
	\begin{figure}
		\centering
		\includegraphics[width=0.8\linewidth]{figs/intro_kde_mirror_2}
	\end{figure}
\end{frame}

\begin{frame}[fragile, t]{Bounded domains}
	A simple trick to overcome bias at boundaries is to mirror the data.
	This ensures that $\hat{f}'(x) = 0$ at the boundary.
	\vfill
	\begin{figure}
		\centering
		\includegraphics[width=0.8\linewidth]{figs/intro_kde_mirror_3}
	\end{figure}
\end{frame}

\begin{frame}[fragile, t]{Bounded domains}
	A simple trick to overcome bias at boundaries is to mirror the data.
	This ensures that $\hat{f}'(x) = 0$ at the boundary.
	\vfill
	\begin{figure}
		\centering
		\includegraphics[width=0.8\linewidth]{figs/intro_kde_mirror_4}
	\end{figure}
\end{frame}

\begin{frame}[fragile, t]{Bounded domains}
	A simple trick to overcome bias at boundaries is to mirror the data.
	This ensures that $\hat{f}'(x) = 0$ at the boundary.
	\vfill
	\begin{figure}
		\centering
		\includegraphics[width=0.8\linewidth]{figs/intro_kde_mirror_5}
	\end{figure}
\end{frame}

\begin{frame}[fragile, t]{Bounded domains}
	A simple trick to overcome bias at boundaries is to mirror the data.
	This ensures that $\hat{f}'(x) = 0$ at the boundary.
	\vfill
	\begin{figure}
		\centering
		\includegraphics[width=0.8\linewidth]{figs/intro_kde_mirror_6}
	\end{figure}
\end{frame}


\section{Extension to $d$ dimensions}


% -------------------------------------------------------------------------
\begin{frame}[fragile, t]{Kernels in 2D}
	An approach to $d$-dimensional estimates is to  write
	\begin{equation*}
	\hat{f}(x) = \frac{1}{h^d} \sum_{i=1}^{N} w_i K \left(\frac{\left \| x - x_i \right \|_p}{h}\right), \text{ where } \sum_{i=1}^{N} w_i = 1.
	\end{equation*}
	\vfill
	\begin{figure}
		\centering
		\includegraphics[width=0.8\linewidth]{figs/2D_intro}
	\end{figure}
\end{frame}


% -------------------------------------------------------------------------
\begin{frame}[fragile, t]{The effect of norms}
	The choice of norm comes in to play when $d \geq 2$, the $p$-norm is 
	\begin{equation*}
		\left\| x \right\| _p := \bigg( \sum_{i=1} \left| x_i \right| ^p \bigg) ^{1/p}.
	\end{equation*}
	\vfill
	\begin{figure}
		\centering
		\includegraphics[width=1\linewidth]{figs/2D_effect_of_norms_tri}
	\end{figure}
\end{frame}


% -------------------------------------------------------------------------
\begin{frame}[fragile, t]{The effect of norms}
	The shape of kernel functions in higher dimensions depend on the value of $p$ in the $p$ norm.
	\vfill
	\begin{figure}
		\centering
		\includegraphics[width=0.8\linewidth]{figs/2D_intro_norms_box}
	\end{figure}
\end{frame}

\begin{frame}[fragile, t]{The effect of norms}
	The shape of kernel functions in higher dimensions depend on the value of $p$ in the $p$ norm.
	\vfill
	\begin{figure}
		\centering
		\includegraphics[width=0.8\linewidth]{figs/2D_intro_norms_tri}
	\end{figure}
\end{frame}

\begin{frame}[fragile, t]{The effect of norms}
	The shape of kernel functions in higher dimensions depend on the value of $p$ in the $p$ norm.
	\vfill
	\begin{figure}
		\centering
		\includegraphics[width=0.8\linewidth]{figs/2D_intro_norms_gaussian}
	\end{figure}
\end{frame}


% -------------------------------------------------------------------------
\begin{frame}[fragile, t]{Example with data}
	As the number of samples grow, the choice of both kernel $K$ and norm $p$ becomes unimportant. The bandwidth $H$ is still important.
	\vfill
	\begin{figure}
		\centering
		\includegraphics[width=1\linewidth]{figs/2D_effect_of_norms_tri_data_1}
	\end{figure}
\end{frame}


\begin{frame}[fragile, t]{Example with data}
	As the number of samples grow, the choice of both kernel $K$ and norm $p$ becomes unimportant. The bandwidth $H$ is still important.
	\vfill
	\begin{figure}
		\centering
		\includegraphics[width=1\linewidth]{figs/2D_effect_of_norms_tri_data_2}
	\end{figure}
\end{frame}


\begin{frame}[fragile, t]{Example with data}
	As the number of samples grow, the choice of both kernel $K$ and norm $p$ becomes unimportant. The bandwidth $H$ is still important.
	\vfill
	\begin{figure}
		\centering
		\includegraphics[width=1\linewidth]{figs/2D_effect_of_norms_tri_data_3}
	\end{figure}
\end{frame}


\begin{frame}[fragile, t]{Example with data}
	As the number of samples grow, the choice of both kernel $K$ and norm $p$ becomes unimportant. The bandwidth $H$ is still important.
	\vfill
	\begin{figure}
		\centering
		\includegraphics[width=1\linewidth]{figs/2D_effect_of_norms_tri_data_4}
	\end{figure}
\end{frame}


\begin{frame}[fragile, t]{Example with data}
	As the number of samples grow, the choice of both kernel $K$ and norm $p$ becomes unimportant. The bandwidth $H$ is still important.
	\vfill
	\begin{figure}
		\centering
		\includegraphics[width=1\linewidth]{figs/2D_effect_of_norms_tri_data_5}
	\end{figure}
\end{frame}


\begin{frame}[fragile, t]{Example with data}
	As the number of samples grow, the choice of both kernel $K$ and norm $p$ becomes unimportant. The bandwidth $H$ is still important.
	\vfill
	\begin{figure}
		\centering
		\includegraphics[width=1\linewidth]{figs/2D_effect_of_norms_tri_data_6}
	\end{figure}
\end{frame}


\begin{frame}[fragile, t]{Example with data}
	As the number of samples grow, the choice of both kernel $K$ and norm $p$ becomes unimportant. The bandwidth $H$ is still important.
	\vfill
	\begin{figure}
		\centering
		\includegraphics[width=1\linewidth]{figs/2D_effect_of_norms_tri_data_7}
	\end{figure}
\end{frame}


\begin{frame}[fragile, t]{Example with data}
	As the number of samples grow, the choice of both kernel $K$ and norm $p$ becomes unimportant. The bandwidth $H$ is still important.
	\vfill
	\begin{figure}
		\centering
		\includegraphics[width=1\linewidth]{figs/2D_effect_of_norms_tri_data_8}
	\end{figure}
\end{frame}


\begin{frame}[fragile, t]{Example with data}
	As the number of samples grow, the choice of both kernel $K$ and norm $p$ becomes unimportant. The bandwidth $H$ is still important.
	\vfill
	\begin{figure}
		\centering
		\includegraphics[width=1\linewidth]{figs/2D_effect_of_norms_tri_data_9}
	\end{figure}
\end{frame}


\begin{frame}[fragile, t]{Example with data}
	As the number of samples grow, the choice of both kernel $K$ and norm $p$ becomes unimportant. The bandwidth $H$ is still important.
	\vfill
	\begin{figure}
		\centering
		\includegraphics[width=1\linewidth]{figs/2D_effect_of_norms_tri_data_10}
	\end{figure}
\end{frame}


\begin{frame}[fragile, t]{Example with data}
	As the number of samples grow, the choice of both kernel $K$ and norm $p$ becomes unimportant. The bandwidth $H$ is still important.
	\vfill
	\begin{figure}
		\centering
		\includegraphics[width=1\linewidth]{figs/2D_effect_of_norms_tri_data_11}
	\end{figure}
\end{frame}


\begin{frame}[fragile, t]{Example with data}
	As the number of samples grow, the choice of both kernel $K$ and norm $p$ becomes unimportant. The bandwidth $H$ is still important.
	\vfill
	\begin{figure}
		\centering
		\includegraphics[width=1\linewidth]{figs/2D_effect_of_norms_tri_data_12}
	\end{figure}
\end{frame}


\begin{frame}[fragile, t]{Example with data}
	As the number of samples grow, the choice of both kernel $K$ and norm $p$ becomes unimportant. The bandwidth $H$ is still important.
	\vfill
	\begin{figure}
		\centering
		\includegraphics[width=1\linewidth]{figs/2D_effect_of_norms_tri_data_13}
	\end{figure}
\end{frame}


\begin{frame}[fragile, t]{Example with data}
	As the number of samples grow, the choice of both kernel $K$ and norm $p$ becomes unimportant. The bandwidth $H$ is still important.
	\vfill
	\begin{figure}
		\centering
		\includegraphics[width=1\linewidth]{figs/2D_effect_of_norms_tri_data_14}
	\end{figure}
\end{frame}


\begin{frame}[fragile, t]{Example with data}
	As the number of samples grow, the choice of both kernel $K$ and norm $p$ becomes unimportant. The bandwidth $H$ is still important.
	\vfill
	\begin{figure}
		\centering
		\includegraphics[width=1\linewidth]{figs/2D_effect_of_norms_tri_data_15}
	\end{figure}
\end{frame}


\begin{frame}[fragile, t]{Example with data}
	As the number of samples grow, the choice of both kernel $K$ and norm $p$ becomes unimportant. The bandwidth $H$ is still important.
	\vfill
	\begin{figure}
		\centering
		\includegraphics[width=1\linewidth]{figs/2D_effect_of_norms_tri_data_16}
	\end{figure}
\end{frame}


\begin{frame}[fragile, t]{Example with data}
	As the number of samples grow, the choice of both kernel $K$ and norm $p$ becomes unimportant. The bandwidth $H$ is still important.
	\vfill
	\begin{figure}
		\centering
		\includegraphics[width=1\linewidth]{figs/2D_effect_of_norms_tri_data_17}
	\end{figure}
\end{frame}


\begin{frame}[fragile, t]{Example with data}
	As the number of samples grow, the choice of both kernel $K$ and norm $p$ becomes unimportant. The bandwidth $H$ is still important.
	\vfill
	\begin{figure}
		\centering
		\includegraphics[width=1\linewidth]{figs/2D_effect_of_norms_tri_data_18}
	\end{figure}
\end{frame}



\section{A fast algorithm}


% -------------------------------------------------------------------------
\begin{frame}[fragile, t]{Linear binning}
	Go through $N$ data points and assign weights to $n$ equidistant grid points. The algorithm runs in $\mathcal{O}(N2^d)$ time in $d$ dimensions.
	\vfill
	\begin{figure}
		\centering
		\includegraphics[width=1\linewidth]{figs/linbin_0}
	\end{figure}
\end{frame}


\begin{frame}[fragile, t]{Linear binning}
	Go through $N$ data points and assign weights to $n$ equidistant grid points. The algorithm runs in $\mathcal{O}(N2^d)$ time in $d$ dimensions.
	\vfill
	\begin{figure}
		\centering
		\includegraphics[width=1\linewidth]{figs/linbin_1}
	\end{figure}
\end{frame}


\begin{frame}[fragile, t]{Linear binning}
	Go through $N$ data points and assign weights to $n$ equidistant grid points. The algorithm runs in $\mathcal{O}(N2^d)$ time in $d$ dimensions.
	\vfill
	\begin{figure}
		\centering
		\includegraphics[width=1\linewidth]{figs/linbin_2}
	\end{figure}
\end{frame}


\begin{frame}[fragile, t]{Linear binning}
	Go through $N$ data points and assign weights to $n$ equidistant grid points. The algorithm runs in $\mathcal{O}(N2^d)$ time in $d$ dimensions.
	\vfill
	\begin{figure}
		\centering
		\includegraphics[width=1\linewidth]{figs/linbin_3}
	\end{figure}
\end{frame}


\begin{frame}[fragile, t]{Linear binning}
	Go through $N$ data points and assign weights to $n$ equidistant grid points. The algorithm runs in $\mathcal{O}(N2^d)$ time in $d$ dimensions.
	\vfill
	\begin{figure}
		\centering
		\includegraphics[width=1\linewidth]{figs/linbin_4}
	\end{figure}
\end{frame}


\begin{frame}[fragile, t]{Linear binning}
	Go through $N$ data points and assign weights to $n$ equidistant grid points. The algorithm runs in $\mathcal{O}(N2^d)$ time in $d$ dimensions.
	\vfill
	\begin{figure}
		\centering
		\includegraphics[width=1\linewidth]{figs/linbin_5}
	\end{figure}
\end{frame}


\begin{frame}[fragile, t]{Linear binning}
	Go through $N$ data points and assign weights to $n$ equidistant grid points. The algorithm runs in $\mathcal{O}(N2^d)$ time in $d$ dimensions.
	\vfill
	\begin{figure}
		\centering
		\includegraphics[width=1\linewidth]{figs/linbin_6}
	\end{figure}
\end{frame}


\begin{frame}[fragile, t]{Linear binning}
	Go through $N$ data points and assign weights to $n$ equidistant grid points. The algorithm runs in $\mathcal{O}(N2^d)$ time in $d$ dimensions.
	\vfill
	\begin{figure}
		\centering
		\includegraphics[width=1\linewidth]{figs/linbin_7}
	\end{figure}
\end{frame}


\begin{frame}[fragile, t]{Linear binning}
	Go through $N$ data points and assign weights to $n$ equidistant grid points. The algorithm runs in $\mathcal{O}(N2^d)$ time in $d$ dimensions.
	\vfill
	\begin{figure}
		\centering
		\includegraphics[width=1\linewidth]{figs/linbin_8}
	\end{figure}
\end{frame}


% -------------------------------------------------------------------------
\begin{frame}[fragile, t]{Sample the kernel}
	Sample the kernel function $K$ at equidistant points.
	The $n$ binned data points and the kernel are then convolved, this runs in $\mathcal{O}(n \log n)$ time, for a total time of $\mathcal{O}(N2^d + n \log n)$.
	\vfill
	\begin{figure}
		\centering
		\includegraphics[width=1\linewidth]{figs/linbin_sample_1}
	\end{figure}
\end{frame}

\begin{frame}[fragile, t]{Sample the kernel}
	Sample the kernel function $K$ at equidistant points.
	The $n$ binned data points and the kernel are then convolved, this runs in $\mathcal{O}(n \log n)$ time, for a total time of $\mathcal{O}(N2^d + n \log n)$.
	\vfill
	\begin{figure}
		\centering
		\includegraphics[width=1\linewidth]{figs/linbin_sample_2}
	\end{figure}
\end{frame}



% -------------------------------------------------------------------------
\begin{frame}[fragile, t]{Linear binning in higher dimensions}
	The extension to $d$ dimensions is relatively straightforward.
	\vfill
	\begin{figure}
		\centering
		\includegraphics[width=1\linewidth]{figs/linbin_2D_1}
	\end{figure}
\end{frame}

\begin{frame}[fragile, t]{Linear binning in higher dimensions}
	The extension to $d$ dimensions is relatively straightforward.
	\vfill
	\begin{figure}
		\centering
		\includegraphics[width=1\linewidth]{figs/linbin_2D_2}
	\end{figure}
\end{frame}

\begin{frame}[fragile, t]{Linear binning in higher dimensions}
	The extension to $d$ dimensions is relatively straightforward.
	\vfill
	\begin{figure}
		\centering
		\includegraphics[width=1\linewidth]{figs/linbin_2D_3}
	\end{figure}
\end{frame}

\begin{frame}[fragile, t]{Linear binning in higher dimensions}
	The extension to $d$ dimensions is relatively straightforward.
	\vfill
	\begin{figure}
		\centering
		\includegraphics[width=1\linewidth]{figs/linbin_2D_4}
	\end{figure}
\end{frame}

\begin{frame}[fragile, t]{Linear binning in higher dimensions}
	The extension to $d$ dimensions is relatively straightforward.
	\vfill
	\begin{figure}
		\centering
		\includegraphics[width=1\linewidth]{figs/linbin_2D_5}
	\end{figure}
\end{frame}

\begin{frame}[fragile, t]{Linear binning in higher dimensions}
	The extension to $d$ dimensions is relatively straightforward.
	\vfill
	\begin{figure}
		\centering
		\includegraphics[width=1\linewidth]{figs/linbin_2D_6}
	\end{figure}
\end{frame}

\begin{frame}[fragile, t]{Linear binning in higher dimensions}
	The extension to $d$ dimensions is relatively straightforward.
	\vfill
	\begin{figure}
		\centering
		\includegraphics[width=1\linewidth]{figs/linbin_2D_7}
	\end{figure}
\end{frame}


\section{KDEpy}


% -------------------------------------------------------------------------
\begin{frame}[fragile, t]{KDEpy}
	If you're interested in KDE in Python, I've written a library.
	\vspace{1em}
	\begin{easylist}[itemize]
		\ListProperties(Space=\listSpace, Space*=\listSpace)
		# GitHub: \url{https://github.com/tommyod/KDEpy}
	\end{easylist}
	\vfill
	\begin{figure}
		\centering
		\includegraphics[width=0.9\linewidth]{figs/profiling_epa}
	\end{figure}
\end{frame}


% -------------------------------------------------------------------------
\begin{frame}[fragile, t]{References}
	References for further reading.
	\vspace{1em}
	\begin{easylist}[itemize]
		\ListProperties(Space=\listSpace, Space*=\listSpace)
		# Silverman, B. W. \emph{Density Estimation for Statistics and Data Analysis}. Chapman and Hall, 1986.
		# Wand, M. P., and M. C. Jones. \emph{Kernel Smoothing.} Chapman and Hall, 1995.
		# Jake VanderPlas. \emph{Kernel Density Estimation in Python.} 2013 \\ \url{https://jakevdp.github.io/blog/2013/12/01/kernel-density-estimation/}
	\end{easylist}
\end{frame}

\end{document}
